\documentclass[10pt, twoside, a4paper]{article}
\usepackage{fancyhdr}
\usepackage{amsmath, amsthm, amssymb}
\usepackage[catalan]{babel}
\usepackage[titles]{tocloft}
\usepackage[utf8]{inputenc}
\usepackage[left=2.15cm, right=2.15cm, top=30mm, bottom=20mm]{geometry}
\usepackage{parskip}
\usepackage{subcaption}
\usepackage{titlesec}
\usepackage{bookmark}
\usepackage{multirow}
\usepackage{graphicx}
\usepackage{physics}
\usepackage{hyperref}
\usepackage{float}
\usepackage{caption}
\captionsetup{labelfont=bf}
\begin{document}

\begin{titlepage}
\centering
{\LARGE Mètodes Numèrics II \par}
\vspace{2cm}
{\Huge \textbf{Pràctica de simulació:} \par}
\vspace{1cm}
{\Huge \textbf{TITOLTITOLTITOL} \par}
\vspace{3cm}
{\Large G01 \par}
\vspace{0.5cm}
{\Large 1549086: Bujones Umbert, Jun Shan\\1666739: Franco Avilés, Eric\\  1672980: González Barea, Eric\\1644841: Vilarrúbias Morral, Natàlia \par}
\vspace{2cm}
{\Large Gener 2025 \par}
\vspace{2cm}
\includegraphics[width=0.4\textwidth]{Logo_UAB.png}


\end{titlepage}

\pagenumbering{gobble}
\renewcommand{\cftsecfont}{}
\renewcommand{\cftsecpagefont}{}
\renewcommand{\cftsecleader}{\cftdotfill{\cftdotsep}}
\renewcommand{\cftdotsep}{0.2}
\setlength{\cftbeforesecskip}{0.5em}
\setlength{\cftbeforesubsecskip}{0.5em}
\tableofcontents

\newpage
\pagenumbering{arabic}
\setcounter{page}{1}

\pagestyle{fancy}
\lhead{\textbf{Pràctica de Simulació}}
\rhead{\textbf{Mètodes Numèrics II}}

\section{Introducció i Objectius}
Amb el pretext de poder calcular l'energia subministrada per una placa solar de superfície igual a 2 m$^3$ situada en un habitatge unifamiliar de Mont-rós (poble de la província de Lleida, Catalunya), en aquesta pràctica ens plantegem modelitzar el moviment del Sol sobre aquesta localitat en el transcurs de tot un any. 

Per aconseguir això últim, hem fet una simulació del Sistema Solar usant el mètode d'Euler per a diverses discretitzacions temporals, tenint en compte un seguit d'aproximacions fonamentals, entre les quals en destaquem la consideració que tots els planetes es mouen sobre el pla de l'eclíptica i que el Sistema Solar simulat està conformat pels 5 primer planetes del cas real. En els annexos, però, ampliem aquest estudi considerant un cas més pròxim a la realitat i adjuntant, a més a més, un seguit d'animacions que ens permetran visualitzar millor la dinàmica del sistema caracteritzat. 

\section{Plantejament del problema del Sistema Solar}

\subsection{Modelització del Sistema Solar}
Per tal de modelitzar el Sistema Solar, partirem de la Segona Llei de Newton i la igualarem a la Lley de la Gravitació Universal, tot dividint per la massa del cos $p$, $M_p$. Fent això ens queda que l'acceleració a la qual està sotmesa aquest cos és:

\begin{equation}
    \derivative[2]{\mathbf{r}^{(p)}}{t} = - G \left[ \sum_{l \neq p} M_l \frac{(\mathbf{r}^{(p)}-\mathbf{r}^{(l)})}{\abs{\mathbf{r}^{(p)} - \mathbf{r}^{(l)}}^3}\right], \hspace{0.25cm} p = 1, 2, \ldots 
\end{equation}

En aquesta equació el vector $\mathbf{r}^{(p)}$ és el vector posició (marquem els vectors en negreta) del cos $p$ respecte del Sol (de manera que el nostre origen de coordenades serà la posició inicial del Sol), $M_l$ la massa dels cossos diferents a $p$ i $\mathbf{r}^{(l)}$ les seves posicions respecte l'origen. Fixem-nos, doncs, que segons aquesta equació, la força que actua sobre un planeta donat es correspon amb la suma de les forces exercides per tots els cossos del sistema solar sobre ell. 

Simplificarem l'expressió definint $\mathbf{d}_{pl} \equiv (\mathbf{r}^{(p)}-\mathbf{r}^{(l)})$. Així, la darrera equació queda com
\begin{equation}
    \derivative[2]{\mathbf{r}^{(p)}}{t} = - G \left[ \sum_{l \neq p} M_l \frac{\mathbf{d}_{lp}}{\abs{\mathbf{d}_{lp}}^3}\right], \hspace{0.25cm} p = 1, 2, \ldots 
\end{equation}

Per tal de facilitar-ne la ressolució, podem transformar aquesta equació diferencial de segon ordre a la següent

\begin{equation}
    \derivative{\mathbf{v}^{(p)}}{t} = -G \left[ \sum_{l \neq p} M_l\frac{\mathbf{d}_{lp}}{\abs{\mathbf{d}_{lp}}^3}\right], \hspace{0.25cm} p = 1,2 \ldots \label{2}
\end{equation}

De forma que tenim ara un conjunt de $n \cross p$ (on $n$ són les dimensions) equacions diferencials de primer ordre a resoldre. A la pràctica, com que tots els planetes del sistema solar tenen orbites coplanàries (a excepció de Plutó, que no el considerarem), podem assumir que estem davant d'un problema bidimensional, de manera que tindre $2 \cross p$ EDOs de primer ordre a resoldre.

Considerarem un model en què els únics cossos del Sistema Solar són Mercuri, Venus, la Terra, Mart i Júpiter (i, naturalment, el Sol), per tal de simplificar les gràfiques. A més a més, el moviment d'aquests cossos serà al pla de l'eclíptica; no considerarem moviments en l'eix $z$. Es podrà trobar una versió més complexa (amb la modelització de tots els planetes i les corresponents gràfiques per diferents temps finals) al repositori de \textit{GitHub} (veure annex \ref{an:a}). A l'annex \ref{an:b} donem un breu resum d'això últim en forma de material extra.

\subsection{Normalització de les equacions}
Per tal de resoldre el problema minimitzant errors i temps de càlcul cal normalitzar \eqref{2}. Definim unes quantitats característiques del sistema: escollim $M_0 = M_s$ (la massa del Sol) i $d_0 = UA$ (unitat astronòmica), ja que així podrem treballar amb valors propers a la unitat. D'aquí se'n deriva que
\begin{equation*}
    t_0 = \sqrt{\frac{d_0^3}{M_s \cdot G}}
\end{equation*}
Usant tot això, podem arribar fàcilment a 
\begin{equation}
    \derivative[2]{\mathbf{\tilde{r}}^{(p)}}{\tilde{t}} = - \sum_{l \neq p} \frac{\mathbf{d}_{lp}}{\abs{\mathbf{d}_{lp}}^3}, \hspace{0.25cm} p = 1,2 \ldots
\end{equation}
d'on tenim que
\begin{equation}
    \boxed{\derivative{\tilde{v}_i^{(p)}}{\tilde{t}} = - \sum_{l \neq p} M_l \frac{d_{lp,i}}{\abs{\mathbf{d}_{lp}}^3} \hspace{0.25cm} p = 1,2 \ldots, \hspace{0.25cm} i = 1, 2} \label{eq5}
\end{equation}

\subsection{Condicions inicials}
Per tal de conèixer les condicions inicials de cadascun dels cossos del sistema solar modelitzat (això és, $\mathbf{r}^{(p)}$ i $\mathbf{v}^{(p)}$) hem usat la base de dades del \textit{Horizons Ephemeris Service} de la NASA, que podeu trobar a \url{https://ssd.jpl.nasa.gov/horizons/}. Agafem com a punt inicial les posicions i les velocitats el dia 01-01-2025 de tots els cossos rellevants del sistema solar.

\subsection{Mètode numèric i avaluació de l'error}
El mètode numèric utilitzat ha estat el mètode d'Euler (per sistemes EDOs)\footnote{Les corresponents equacions es poden consultar als apunts de l'assignatura.} aplicat a l'equació \eqref{eq5} per diferents discretitzacions temporals: $dt = 1$ any, $dt = 1$ mes i $dt = 1$ dia.

Per tal d'avaluar l'error comés estudiarem l'evolució d'una de les quantitats conservades d'aquest sistema: L'energia total $\varepsilon$. Idealment, a cada cas d'iteració el valor de $E$ ha de ser el mateix que a $t_0$; les fluctuacions en aquestes quantitats ens permetran determinar l'error numèric comès segons:
\begin{align}
    E_{\varepsilon_i} = \left( \frac{\Delta \varepsilon_i}{\varepsilon_0} \right) & = \frac{\varepsilon_i - \varepsilon_0}{\varepsilon_0} 
\end{align}

On el valor de $\varepsilon$ i $L$ es pot treure a partir de les equacions fonamentals de la dinàmica.

\section{Plantejament del problema de la placa solar}

\subsection{Modelització del moviment del Sol sobre Mont-rós}
Per modelitzar el moviment del Sol des d'un sistema de referència situat en una teulada a Mont-rós, partim d'unes coordenades esfèriques on un angle $\theta$ indica l'alçada del sol i un angle $\phi$ indica la posició horizontal del sol utilitzant el centre de la placa solar com a origen de coordenades.

Així doncs, partint d'aquests angles, realitzem una discretització de cada angle en funció de les hores de llum solar que té cada dia de l'any\footnote{Informació que hem extret de \url{https://meteogram.es/sol/espana/vielha/}}. En aquesta discretització tenim en compte els angles $\phi$ que limiten la regió on rebem incidència de llum solar i els angles $\theta$ màxims als quals arriba el sol depenent de l'estació de l'any, aquests últims delimitats en un interval determinat per la latitud de Mont-rós $\pm 15^\circ$.

\textbf{(falta imagen de las coordenadas utilizadas)}

\subsection{Modelització de l'electricitat generada per la placa solar}
A partir del model del moviment solar, plantegem els mateixos angles $\theta$ i $\phi$ però escollint com a origen de coordenades el centre de la placa solar; d'aquesta manera podrem definir la incidència lumínica que rebem en cada discretització temporal com $W = W_{max}\cos(\theta)\cos(\phi)$, on $W_{max} = 1000 \text{ W/m$^2$}$. Amb aquesta expressió ja podem calcular l'electricitat generada en tot l'any tenint en compte la variació horària i del moviment solar en les diferents estacions.

D'altra banda, en el nostre model també tenim en compte l'horizont de Mont-rós (el qual es caracteritza per estar envoltat de muntanyes) mitjançant una variació en els angles $\phi$ inicials i finals, i les possibles variacions climàtiques al llarg de l'any mitjançant un factor $0<K<1$ que varia depenent l'època de l'any.

\textbf{(estaria guay poner una imagen del sistema desde la placa, si se puede)}

\subsection{Avaluació de l'error?}
SI FEU ALGO D'AIXÒ

\section{Resultats i discussió}

\subsection{Sistema Solar}
Després d'implementar el mètode numèric s'han trobat les òrbites (per a cada valor de $dt$) fins a $t=1$ any (terrestre) que es poden veure a la figura \ref{fig1}. Més tard, a la figura \ref{fig4}, podem veure les òrbites per a $t=100$ anys per la discretització $dt=1$ dia amb el corresponent error.
 
\begin{figure}[h]
    \centering
    
    \begin{subfigure}[b]{0.32\linewidth}
        \centering
        \includegraphics[width=\linewidth]{../sist_solar/orbites_euler_1_d1hora.png}
        \caption{Òrbites per $dt=1$ hora.}
        \label{fig:euler_implicit_solucio}
    \end{subfigure}
    \hfill
    \begin{subfigure}[b]{0.32\linewidth}
        \centering
        \includegraphics[width=\linewidth]{../sist_solar/orbites_euler_1_d1dia.png}
        \caption{Òrbites per $dt=1$ dia.}
        \label{fig:euler_implicit_errors}
    \end{subfigure}
    \hfill
    \begin{subfigure}[b]{0.32\linewidth}
        \centering
        \includegraphics[width=\linewidth]{../sist_solar/orbites_euler_1_d1mes.png}
        \caption{Òrbites per $dt=1$ mes.}
        \label{fig:euler_implicit_errors}
    \end{subfigure}
    
    \caption{Òrbites del sistema solar a $t=1$ any (terrestre) per diferents valors de la discretització obtingudes.}
    \label{fig1}
\end{figure}

A l'annex \ref{an:a} podeu trobar les gràfiques corresponents a tot el Sistema Solar (amb els cossos no descrits aquí) i un seguit d'animacions per tal de visualitzar més clarament l'evolució temporal de tots els cossos fins a un $t_{final}=15$ anys.

Els errors relatius en l'energia associats als diferents planetes es poden veure a la figura \ref{fig3}

\begin{figure}[h!]
    \centering
    
    \begin{subfigure}[b]{0.32\linewidth}
        \centering
        \includegraphics[width=\linewidth]{../Error/error_1_hora.png}
        \caption{Errors per $dt=1$ hora.}
    \end{subfigure}
    \hfill
    \begin{subfigure}[b]{0.32\linewidth}
        \centering
        \includegraphics[width=\linewidth]{../Error/error_1_dia.png}
        \caption{Errors per $dt=1$ dia.}
    \end{subfigure}
    \hfill
    \begin{subfigure}[b]{0.32\linewidth}
        \centering
        \includegraphics[width=\linewidth]{../Error/error_1_mes.png}
        \caption{Errors per $dt=1$ mes.}
    \end{subfigure}
    
    \caption{Errors relatius respecte l'energia inicial pels diferents cossos del Sistema Solar a partir de la solució trobada mitjançant el mètode d'Euler, classificat en funció de la discretització emprada, fins a $t_{final}=1$ any.}
    \label{fig3}
\end{figure}

Tal i com podem veure a les anteriors tres gràfiques, la discretització amb un major error numèric és la corresponent a $dt = 1$ mes, degut a que els grans salts en el temps fan que els errors asssociats al mètode d'Euler, que de per si no és molt bo, es disparin. En concret, podem veure com l'error associat a l'energia de Venus s'arriba a disparar fins a més d'un 80\%. La Terra i Mercuri tenen també errors relatius grans, sobretot al principi, essent aquests propers al 20\% (valor absolut). 

De fet, si ens fixem en la figura \ref{fig1}, per la discretització $dt=1$ mes, podem veure que Mercuri surt disparat del Sistema Solar. Que això no es vegi reflectit, aparentment, en l'error associat a l'energia de Mercuri (que es manté pràcticament constant en una desviació del 20\%) pot ser degut a què les variacions en els termes en l'energia cinètica i l'energia potencial d'aquest cos es compensen conforme l'astre s'allunya de la seva posició inicial.

Tant a la discretització de $dt = 1$ dia com a la de $dt = 1$ hora tenim errors associats als planetes més exteriors del sistema modelitzat similars, essent aquests lleugerament menors en el cas de la discretització per hores, a costa d'un temps de càlcul major. La principal diferència rau, però, en l'error associat a Mercuri: per a la discretització en dies presenta un error relatiu molt major al de la resta de cossos (exceptuant el Sol), assolint pics de fins a un 2.5\%; per a la discretització en hores aquest efecte es disminueix significativament.

Cal que comentem a part, però, el cas del Sol Per a les tres discretitzacions presenta una evolució de l'error associat a l'energia molt similar: en tots els casos té una clara tendència a desviar-se negativament. Podem explicar això si pensem en què el Sol és, dels 6 cossos modelitzats, el que ha tenir posicions i velocitats menors, de forma que l'acumulació d'errors associats al mètode numèric pot fer que aquestes petites quantitats es vegin afectades de forma significativa.

Si fem un estudi de l'evolució del Solar fins a un $t_{final}$ major, usant $dt=1$ dia, els resultats que s'obtenen són els que es poden veure a la figura \ref{fig4}.

\begin{figure}[h!]
    \centering
    \begin{subfigure}[b]{0.48\linewidth}
        \centering
        \includegraphics[width=\linewidth]{../sist_solar/orbites_euler_100_d1dia.png}
        \caption{Òrbites fins a $t_{final}=100$ anys amb $dt=1$ dia.}
    \end{subfigure}
    \hfill
    \begin{subfigure}[b]{0.48\linewidth}
        \centering
        \includegraphics[width=\linewidth]{../Error/error_100_dia.png}
        \caption{Errors per $dt=1$ dia.}
    \end{subfigure}
    \caption{Estudi de l'evolució del Sistema Solar a $t_{final}$ major.}
    \label{fig4}
\end{figure}

Ara es posa de manifest el caràcter oscil·latori de l'error relatiu per tots els cossos del Sistema Solar. Podem veure com l'error màxim obtingut és proper al 6.5\% (valor absolut) en el cas del Sol, pels motius ja comentats abans\footnote{Per motius de qualitat, de la gràfica associada a l'error per $t_{final}=100$ anys arriba només fins als 50 anys. Podeu trobar més informació a l'enllaç de l'annex \ref{an:a}.}. La resta de planetes, exceptuant el cas de Mercuri, no arriben en cap cas a errors superiors al 1\%.




\subsection{Moviment del Sol sobre Mont-rós}


\subsection{Energia subministrada per la placa solar}

Factors que hem tingut en compte a l'hora de calcular l'energia elèctrica generada: 

Horitzó no pla
Les muntanyes que envolten el poble tenen un impacte significatiu a les hores de llum diaries. Les muntanyes a l'est, per on surt el Sol, són més pronunciades que les de l'oest, per on es pon. Per aquest motiu, hem exclòs les dades corresponents a angles inferiors a -70° i superiors a 80°, ja que no contribueixen de manera significativa a la irradiació efectiva sobre les plaques.

Meteorologia 
S'ha dut a terme un estudi climàtic basat en dades històriques (he deposar l'enllaç), per predir l'impacte de pluges i nevades sobre el rendiment de les plaques solars:

\textbf{Pluges}

Les precipitacions lleus no només no afecten negativament el rendiment de les plaques, sinó que poden ser beneficioses, ja que netegen la seva superfície. Tanmateix, hem considerat perjudicials:
\begin{itemize}
    \item Pluges moderades (4-10 mm/h): redueixen el rendiment un 15\%
    \item Pluges intenses ($>$ 10 mm/h): redueixen el rendiment un 25\%
\end{itemize}
     
\textbf{Nevades}

Segons les dades històriques, al poble neva entre gener i mitjans de març. Tot i que la neu és altament perjudicial si s'acumula sobre les plaques (bloquejant completament els raigs solars), la inclinació de les plaques (42,5°) redueix considerablement aquest risc. A més, les nevades a la zona són puntuals i de baixa intensitat, cosa que facilita el desglaç natural.
En cas necessari, es podria instal·lar un sistema antidesglaç o una cobertura hidrofòbica, tot i que aquestes opcions consumirien part de l'energia generada. A més, la neu acumulada al terra al voltant de les plaques pot crear un efecte reflectant que incrementa lleugerament la captació solar. Finalment, s'ha aplicat una reducció del 20\% al rendiment durant el període de neu, com a estimació conservadora.

\textbf{Temperatura}

Les plaques fotovoltaiques comencen a perdre eficiència a temperatures superiors als 25 °C, amb una disminució del 0,3\% per grau ( posar enllaç). A Mont-rós, les temperatures màximes s'observen a l'estiu, arribant als 31 °C durant 1-2 hores al migdia (entre les 12:00 i les 15:00). Aquestes temperatures no són prou significatives per afectar de manera rellevant el rendiment global, per la qual cosa s'ha considerat negligible aquest efecte.

Un cop considerats tots els factors mencionats, hem procedit amb el càlcul de l'energia elèctrica generada per la placa fotovoltaica. 

Aquesta informació l'hem comparat amb un model referencial que presenta característiques i ubicació semblants a la nostra placa fotovoltaica, obtingut a partir de la plataforma PVGIS. A les figures següents es poden observar els histogrames corresponents a l'energia elèctrica generada cada mes al llarg d'un any de la nostra placa i la obtinguda a PVGIS. 
\begin{figure}[h!]
    \centering
    \begin{minipage}{0.45\linewidth} 
        \centering
        \includegraphics[width=\linewidth]{Histograma_PVGIS.png}
        \caption{Histograma de l'energia generada durant un any extreta de PVGIS}
        \label{fig:figura1}
    \end{minipage}\hfill 
    \begin{minipage}{0.45\linewidth} 
        \centering
        \includegraphics[width=\linewidth]{../Mov_sol/histograma.png}
        \caption{Histograma de l'energia generada durant un any del nostre model}
        \label{fig:figura2}
    \end{minipage}
\end{figure}


\begin{table}[h!]
    \centering
    \begin{tabular}{|c|c|c|c|c|c|c|c|c|c|c|c|c|}
    \hline
     & Gener & Febrer & Març & Abril & Maig & Juny & Juliol & Agost & Setembre & Octubre & Novembre & Desembre \\
    \hline
    PVGIS & 42,05 & 47,35 & 55,13 & 50,56 & 54,98 & 55,54 & 62,04 & 61,03 & 54,57 & 48,27 & 38,93 & 39,81 \\
    \hline
    Nostre Model & 53,91 & 56,99 & 76,17 & 67,97 & 77,00 & 90,62 & 104,73 & 95,69 & 72,34 & 78,60 & 67,46 & 61,87 \\
    \hline
    \end{tabular}
    \caption{Taula comparativa de l'energia generada durant un any amb el model PVGIS i el nostre. }
    \end{table}
    

\section{Conclusions}

BLAHBLAHBLAH DEL SISTEMA SOLAR

\newpage
\begin{thebibliography}{99}
    \bibitem{ref1}
    Escartín, J.M. i Navau, C. \textit{Mètodes Numèrics II}. Apunts de l'Assignatura (veure CV).

    \bibitem{ref2}
    Aguilar, L. \textit{Modelizando el Sistema Solar}. Consultat el: 11/12/2024. \url{https://www.astrosen.unam.mx/~aguilar/MySite/Teaching_files/BasicEqns-1.pdf}.

    \bibitem{ref3}
    \textit{Horizon Ephemeris}, NASA. Web de consulta de les posicions de tots els astres del sistema solar. Consultat el: 23/12/2024. \url{https://ssd.jpl.nasa.gov/horizons/}.

    \bibitem{ref4}
    Universidad de Granada. \textit{El Sistema Solar y las Galaxias. Una Introducción a la Dinámica Molecular}. Consultat el: 11/12/2024. \url{https://ergodic.ugr.es/cphys/LECCIONES/ssolar/planetas-SLIDES.pdf}.

    \bibitem{ref5}
    \textit{MeteoGram}. Web per a consultar els horaris de les sortides i postes del Sol a tot el planeta. Consultat el: 05/01/2025. \url{https://meteogram.es/sol/espana/vielha/}.

    \bibitem{ref6}
    \textit{NASA Power}, NASA. Web per consultar la metereología de qualsevol lloc del món. Consultat el 07/01/2025. \url{https://power.larc.nasa.gov/data-access-viewer/}.

    
\end{thebibliography}

\newpage
\appendix
{\Huge{\textbf{Annexos}}}
\section{Repositori de \textit{GitHub}}
\label{an:a}
Podeu trobar els codis usats en \textit{Fortran}, les representacions de les gràfiques usant \textit{Gnuplot}, animacions de les simulacions del sistema solar i més al següent repositori (públic) de \textit{GitHub}: \url{https://github.com/elitus7/PSimulacio_MN2}.

\section{Simulació del Sistema Solar amb més cossos}
\label{an:b}
A mode d'extra, afegim en aquest annex els resultats de la simulació del Sistema Solar considerant també Saturn, Urà i Neptú (més enllà dels cossos del model del text principal) mantenint, però, la condició que el moviment es dóna en un únic pla. A la figura \ref{fig:an1} podeu veure les òrbites obtingudes usant una discretització de $dt=1$ dia, que com hem vist és la millor pel que fa a temps de càlcul i exactitud, i un $t_{final}=150$ anys.

\begin{figure}[h]
    \centering
    \includegraphics[width=1.0\linewidth]{../sist_solar/orbites_euler_TOTS_150_d1dia.png}
    \caption{Òrbites obtingudes amb el mètode d'Euler per a una simulació amb $dt=1$ any fins a $t_{final}=150$ anys per tots els cossos importants del Sistema Solar.}
    \label{fig:an1}
\end{figure}

Al repositori de \textit{GitHub} podreu veure una animació, a $t_{final}=15$ anys, en la que es poden veure les òrbites anteriors.
\end{document}