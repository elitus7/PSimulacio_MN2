\documentclass[10pt, twoside]{article}
\usepackage{fancyhdr}
\usepackage{amsmath, amsthm, amssymb}
\usepackage[catalan]{babel}
\usepackage[titles]{tocloft}
\usepackage[utf8]{inputenc}
\usepackage[left=2.15cm, right=2.15cm, top=30mm, bottom=20mm]{geometry}
\usepackage{parskip}
\usepackage{titlesec}
\usepackage{bookmark}
\usepackage{multirow}
\usepackage{graphicx}
\usepackage{physics}
\usepackage{hyperref}
\usepackage{float}
\usepackage{caption}
\captionsetup{labelfont=bf}
\begin{document}

\begin{titlepage}
\centering
{\LARGE Mètodes Numèrics II \par}
\vspace{2cm}
{\Huge \textbf{Pràctica de simulació:} \par}
\vspace{1cm}
{\Huge \textbf{TITOLTITOLTITOL} \par}
\vspace{3cm}
{\Large G01 \par}
\vspace{0.5cm}
{\Large 1548086: Bujones Umbert, Jun Shan\\1666739: Franco Avilés, Eric\\  1672980: González Barea, Eric\\1644841: Vilarrúbias Morral, Natàlia \par}
\vspace{2cm}
{\Large Gener 2025 \par}
\vspace{2cm}
\includegraphics[width=0.4\textwidth]{Logo_UAB.png}


\end{titlepage}

\pagenumbering{gobble}
\renewcommand{\cftsecfont}{}
\renewcommand{\cftsecpagefont}{}
\renewcommand{\cftsecleader}{\cftdotfill{\cftdotsep}}
\renewcommand{\cftdotsep}{0.2}
\setlength{\cftbeforesecskip}{0.5em}
\setlength{\cftbeforesubsecskip}{0.5em}
\tableofcontents

\newpage
\pagenumbering{arabic}
\setcounter{page}{1}

\pagestyle{fancy}
\lhead{\textbf{Pràctica de Simulació}}
\rhead{\textbf{Mètodes Numèrics II}}

\section{Introducció}
Here goes \textit{blahblahblah}

\section{Plantejament del problema del Sistema Solar}
Here goes more \textit{blahblahblah}  AAAAAAA

\subsection{Modelització del Sistema Solar}
Per tal de modelitzar el Sistema Solar, partirem de la Segona Llei de Newton i la igualarem a la Lley de la Gravitació Universal, tot dividint per la massa del planeta $p$, $M_p$. Fent això ens queda que l'acceleració a la qual està sotmesa aquest planeta és:

\begin{equation}
    \derivative[2]{\mathbf{r}^{(p)}}{t} = - G \left[ M_s \frac{\mathbf{r}^{(p)}}{\abs{
    \mathbf{r}^{(p)}}^3} + \sum_{l \neq p} M_l \frac{(\mathbf{r}^{(p)}-\mathbf{r}^{(l)})}{\abs{\mathbf{r}^{(p)} - \mathbf{r}^{(l)}}^3}\right], \hspace{0.25cm} p = 1, 2, \ldots 
\end{equation}

A aquesta equació el vector $\mathbf{r}^{(p)}$ és el vector posició (marquem els vectors en negreta) del cos $p$ respecte del Sol (que el situem a l'origen de coordenades), $M_s$ és la massa del Sol, $M_l$ la massa dels cossos diferents a $p$ i $\mathbf{r}^{(l)}$ les seves posicions respecte l'origen. Fixem-nos, doncs, que segons aquesta equació, la força que actua sobre un planeta donat es correspon amb la suma de les forces exercides per tots els cossos del sistema solar sobre ell. 

Per tal de facilitar-ne la ressolució, podem transformar aquesta equació diferencial de segon ordre a la següent

\begin{equation}
    \derivative{v_i^{(p)}}{t} = -G \left[ M_s \frac{r_i^{(p)}}{\abs{
        \mathbf{r}^{(p)}}^3} + \sum_{l \neq p} M_l\frac{(r_i^{(p)}-r_i^{(l)})}{\abs{\mathbf{r}^{(p)} - \mathbf{r}^{(l)}}^3} \right], \hspace{0.25cm} p = 1,2 \ldots, \hspace{0.25cm} i = 1,2,3. \label{2}
\end{equation}

on hem usat que

\begin{equation}
    v_i^{(p)} \equiv \derivative{r_i^{(p)}}{t}, \hspace{0.25cm} p = 1,2 \ldots, \hspace{0.25cm} i = 1,2,3.
\end{equation}
De forma que tenim ara un conjunt de $3 \cross p$ equacions diferencials de primer ordre a resoldre.

\subsection{Normalització de les equacions}
Per tal de resoldre el problema minimitzant errors i temps de càlcul cal normalitzar \eqref{2}. Definim unes quantitats característiques del sistema: escollim $M_0 = M_s$ i $d_0 = UA$ (unitat astronòmica), ja que així podrem treballar amb valors propers a la unitat.
\end{document}